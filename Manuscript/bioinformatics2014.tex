\documentclass[a4paper,twoside]{article}

\usepackage{epsfig}
\usepackage{subfigure}
\usepackage{calc}
\usepackage{amssymb}
\usepackage{amstext}
\usepackage{amsmath}
\usepackage{amsthm}
\usepackage{multicol}
\usepackage{pslatex}
\usepackage{apalike}
\usepackage{SCITEPRESS}
\usepackage[small]{caption}

\subfigtopskip=0pt
\subfigcapskip=0pt
\subfigbottomskip=0pt

\begin{document}

\title{Authors' Instructions  \subtitle{Preparation of Camera-Ready Contributions to SCITEPRESS Proceedings} }

\author{\authorname{First Author Name\sup{1}, Second Author Name\sup{1} and Third Author Name\sup{2}}
\affiliation{\sup{1}Institute of Problem Solving, XYZ University, My Street, MyTown, MyCountry}
\affiliation{\sup{2}Department of Computing, Main University, MySecondTown, MyCountry}
\email{\{f\_author, s\_author\}@ips.xyz.edu, t\_author@dc.mu.edu}
}

\keywords{}

\abstract}

\onecolumn \maketitle \normalsize \vfill

\section{\uppercase{Introduction}}
\label{sec:introduction}

\noindent Centralized massively parallel nucleic acid sequencing has led to an exponential increase in genomic data generation that threatens to outpace advances in data storage and analysis \cite{kahn_future_2011,teeling_current_2012}. In addition, small distributed sequencing platforms such as the IonTorrent and MiSeq have emerged that promise to provide point of care / investigation capabilities with near-real time generation of genomic data \cite{loman_performance_2012}. This capability will allow the research community to rapidly disseminate data, especially where decisions may be time-critical; e.g., in clinical medicine and epidemiological investigations. Better algorithms, more powerful analytical tools and state-of-the art infrastructure are needed to analyze these datasets in near-real time, store the raw and computed data and provide the essential biological information to a wide range of end-users, including those with little or no training in bioinformatics, in readily understandable and useful formats.

Efforts to simplify bioinformatics workflows such as Taverna \cite{lanzen_taverna_2008} and Galaxy \cite{goecks_galaxy:_2010} have been created, and provide an effective means for users to create bioinformatics workflows. However, data is not integrated with these tools, requiring transfer of genomic sequences from public or private databases, the re-computation of analyses and the inability to compare thousands of genomes. Likewise, online repositories of genomic sequence data such as the National Center for Biotechnology Information (\url{http://www.ncbi.nlm.nih.gov/}) and the Genomes Online Database (\url{http://www.genomesonline.org/cgi-bin/GOLD/}) provide a wealth of data, but are decoupled from an efficient analysis platform. Additionally, storage and computational analysis of thousands of genomes has moved beyond the standard desktop computer, and even with more memory, efficient methodologies and algorithms, servers storing and analyzing thousands of genomic sequences require leading-edge hardware, and the ability to scale in order to meet the computational requirements projected by this increase in data.

We have previously designed Panseq, a suite of software tools called for the automated comparison of multiple genomes \cite{laing_pan-genome_2010,laing_identification_2011}. Panseq is based on the concept of the bacterial pan-genome, the outputs of which enhance our understanding of the evolution of specific bacterial groups, and the genetic basis of important phenotypic traits which differ among these groups.

In this study, we hope to transform the way the research community analyzes genomic data. In light of the torrent of sequencing data that has been generated and the uptake of distributed sequencing technologies that can provide near-real time data, we will create a counterpart computational platform that can provide near-real time analyses. We will provide all publicly available data for species of interest, pre-computed analyses of the data and novel analyses tools that will decrease computation times and will allow rapid assessment of user-uploaded data. The web-interface to this computational platform will obviate the need for command-line skills, or a particular computer environment. As more of the research community uses the platform, the number of genomic sequences it has processed / analyzed will increase, adding further value to the platform and thus will attract more users. From the outset feedback will be solicited from users to improve performance of the entire system and the needs of specific user groups.   Once completed, the platform will provide data outputs that serve a wide community of users, with translation of the genomic data into biologically relevant reports, useful to researchers in a variety of disciplines.


1. We will create a broadly accessible, integrated platform for comparative genomic data storage and analyses

2. The platform will provide near-real time analysis of thousands of genomic sequences using novel computational approaches

3. The platform will generate results that are understandable and useful to a wide community of researchers

\section{\uppercase{Conclusions}}
\label{sec:conclusion}

\noindent Please note that ONLY the files required to compile your paper should be submitted. Previous versions or examples MUST be removed from the compilation directory before submission.

We hope you find the information in this template useful in the preparation of your submission.

\section*{\uppercase{Acknowledgements}}

\noindent If any, should be placed before the references section
without numbering. To do so please use the following command:
\textit{$\backslash$section*\{ACKNOWLEDGEMENTS\}}


\vfill
\bibliographystyle{apalike}
{\small
\bibliography{example}}


\section*{\uppercase{Appendix}}

\noindent If any, the appendix should appear directly after the
references without numbering, and not on a new page. To do so please use the following command:
\textit{$\backslash$section*\{APPENDIX\}}

\vfill
\end{document}

