\documentclass[a4paper,twoside]{article}

\usepackage{epsfig}
\usepackage{subfigure}
\usepackage{calc}
\usepackage{amssymb}
\usepackage{amstext}
\usepackage{amsmath}
\usepackage{amsthm}
\usepackage{multicol}
\usepackage{pslatex}
\usepackage{apalike}
\usepackage{SCITEPRESS}
\usepackage[small]{caption}

\subfigtopskip=0pt
\subfigcapskip=0pt
\subfigbottomskip=0pt

\begin{document}

\title{Authors' Instructions  \subtitle{Preparation of Camera-Ready Contributions to SCITEPRESS Proceedings} }

\author{\authorname{First Author Name\sup{1}, Second Author Name\sup{1} and Third Author Name\sup{2}}
\affiliation{\sup{1}Institute of Problem Solving, XYZ University, My Street, MyTown, MyCountry}
\affiliation{\sup{2}Department of Computing, Main University, MySecondTown, MyCountry}
\email{\{f\_author, s\_author\}@ips.xyz.edu, t\_author@dc.mu.edu}
}

\keywords{The paper must have at least one keyword. The text must be set to 9-point font size and without the use of bold or italic font style. For more than one keyword, please use a comma as a separator. Keywords must be titlecased.}

\abstract{The abstract should summarize the contents of the paper and should contain at least 70 and at most 200 words. The text must be set to 9-point font size.}

\onecolumn \maketitle \normalsize \vfill

\section{\uppercase{Introduction}}
\label{sec:introduction}

\noindent Your paper will be part of the conference proceedings
therefore we ask that authors follow the guidelines explained in
this example in order to achieve the highest quality possible
\cite{Smith98}.

Be advised that papers in a technically unsuitable form will be
returned for retyping. After returned the manuscript must be
appropriately modified.

\section{\uppercase{Manuscript Preparation}}

\noindent We strongly encourage authors to use this document for the
preparation of the camera-ready. Please follow the instructions
closely in order to make the volume look as uniform as possible
\cite{Moore99}.

Please remember that all the papers must be in English and without
orthographic errors.

Do not add any text to the headers (do not set running heads) and
footers, not even page numbers, because text will be added
electronically.

For a best viewing experience the used font must be Times New
Roman, except on special occasions, such as program code
\ref{subsubsec:program_code}.


\subsection{Manuscript Setup}

\noindent The template is composed by a set of 7 files, in the
following 2 groups:\\
\noindent {\bf Group 1.} To format your paper you will need to copy
into your working directory, but NOT edit, the following 4 files:
\begin{verbatim}
  - apalike.bst
  - apalike.sty
  - article.cls
  - scitepress.sty
\end{verbatim}

\noindent {\bf Group 2.} Additionally, you may wish to copy and edit
the following 3 example files:
\begin{verbatim}
  - example.bib
  - example.tex
  - scitepress.eps
\end{verbatim}


\subsection{Page Setup}

The paper size must be set to A4 (210x297 mm). The document
margins must be the following:

\begin{itemize}
    \item Top: 3,3 cm;
    \item Bottom: 4,2 cm;
    \item Left: 2,6 cm;
    \item Right: 2,6 cm.
\end{itemize}

It is advisable to keep all the given values because any text or
material outside the aforementioned margins will not be printed.

\subsection{First Section}

This section must be in one column.

\vfill
\subsubsection{Title and Subtitle}

Use the command \textit{$\backslash$title} and follow the given structure in "example.tex". The title and subtitle must be with initial letters
capitalized (titlecased). If no subtitle is required, please remove the corresponding \textit{$\backslash$subtitle} command. In the title or subtitle, words like "is", "or", "then", etc. should not be capitalized unless they are the first word of the subtitle. No formulas or special characters of any form or language are allowed in the title or subtitle.

\subsubsection{Authors and Affiliations}

Use the command \textit{$\backslash$author} and follow the given structure in "example.tex".

\subsubsection{Keywords}

Use the command \textit{$\backslash$keywords} and follow the given structure in "example.tex". Each paper must have at least one keyword. If more than one is specified, please use a comma as a separator. The sentence must end with a period.

\subsubsection{Abstract}

Use the command \textit{$\backslash$abstract} and follow the given structure in "example.tex".
Each paper must have an abstract up to 200 words. The sentence
must end with a period.

\subsection{Introduction}
\noindent Centralized massively parallel nucleic acid sequencing has led to an exponential increase in genomic data generation that threatens to outpace advances in data storage and analysis \cite{kahn_future_2011,teeling_current_2012}. In addition, small distributed sequencing platforms such as the IonTorrent and MiSeq have emerged that promise to provide point of care / investigation capabilities with near-real time generation of genomic data \cite{loman_performance_2012}. This capability will allow the research community to rapidly disseminate data, especially where decisions may be time-critical; e.g., in clinical medicine and epidemiological investigations. Better algorithms, more powerful analytical tools and state-of-the art infrastructure are needed to analyze these datasets in near-real time, store the raw and computed data and provide the essential biological information to a wide range of end-users, including those with little or no training in bioinformatics, in readily understandable and useful formats.

Efforts to simplify bioinformatics workflows such as Taverna \cite{lanzen_taverna_2008} and Galaxy \cite{goecks_galaxy:_2010} have been created, and provide an effective means for users to create bioinformatics workflows. However, data is not integrated with these tools, requiring transfer of genomic sequences from public or private databases, the re-computation of analyses and the inability to compare thousands of genomes. Likewise, online repositories of genomic sequence data such as the National Center for Biotechnology Information (\url{http://www.ncbi.nlm.nih.gov/}) and the Genomes Online Database (\url{http://www.genomesonline.org/cgi-bin/GOLD/}) provide a wealth of data, but are decoupled from an efficient analysis platform. Additionally, storage and computational analysis of thousands of genomes has moved beyond the standard desktop computer, and even with more memory, efficient methodologies and algorithms, servers storing and analyzing thousands of genomic sequences require leading-edge hardware, and the ability to scale in order to meet the computational requirements projected by this increase in data.

We have previously designed Panseq, a suite of software tools called for the automated comparison of multiple genomes \cite{laing_pan-genome_2010,laing_identification_2011}. Panseq is based on the concept of the bacterial pan-genome, the outputs of which enhance our understanding of the evolution of specific bacterial groups, and the genetic basis of important phenotypic traits which differ among these groups.

In this study, we hope to transform the way the research community analyzes genomic data. In light of the torrent of sequencing data that has been generated and the uptake of distributed sequencing technologies that can provide near-real time data, we will create a counterpart computational platform that can provide near-real time analyses. We will provide all publicly available data for species of interest, pre-computed analyses of the data and novel analyses tools that will decrease computation times and will allow rapid assessment of user-uploaded data. The web-interface to this computational platform will obviate the need for command-line skills, or a particular computer environment. As more of the research community uses the platform, the number of genomic sequences it has processed / analyzed will increase, adding further value to the platform and thus will attract more users. From the outset feedback will be solicited from users to improve performance of the entire system and the needs of specific user groups.   Once completed, the platform will provide data outputs that serve a wide community of users, with translation of the genomic data into biologically relevant reports, useful to researchers in a variety of disciplines.


1. We will create a broadly accessible, integrated platform for comparative genomic data storage and analyses

2. The platform will provide near-real time analysis of thousands of genomic sequences using novel computational approaches

3. The platform will generate results that are understandable and useful to a wide community of researchers





\subsection{Design}







\subsection{Functionality}





\subsubsection{Tables}

Tables must appear inside the designated margins or they may span
the two columns.

Tables in two columns must be positioned at the top or bottom of the
page within the given margins. To span a table in two columns please add an asterisk (*) to the table \textit{begin} and \textit{end} command.

Example: \textit{$\backslash$begin\{table*\}}

\hspace*{1.5cm}\textit{$\backslash$end\{table*\}}\\

Tables should be centered and should always have a caption
positioned above it. The font size to use is 9-point. No bold or
italic font style should be used.

The final sentence of a caption should end with a period.

\begin{table}[h]
\caption{This caption has one line so it is
centered.}\label{tab:example1} \centering
\begin{tabular}{|c|c|}
  \hline
  Example column 1 & Example column 2 \\
  \hline
  Example text 1 & Example text 2 \\
  \hline
\end{tabular}
\end{table}

\begin{table}[h]
\caption{This caption has more than one line so it has to be
justified.}\label{tab:example2} \centering
\begin{tabular}{|c|c|}
  \hline
  Example column 1 & Example column 2 \\
  \hline
  Example text 1 & Example text 2 \\
  \hline
\end{tabular}
\end{table}

Please note that the word "Table" is spelled out.


\subsubsection{Figures}

Please produce your figures electronically, and integrate them into
your document and zip file.

Check that in line drawings, lines are not interrupted and have a
constant width. Grids and details within the figures must be clearly
readable and may not be written one on top of the other.

Figure resolution should be at least 300 dpi.

Figures must appear inside the designated margins or they may span
the two columns.

Figures in two columns must be positioned at the top or bottom of
the page within the given margins. To span a figure in two columns please add an asterisk (*) to the figure \textit{begin} and \textit{end} command.

Example: \textit{$\backslash$begin\{figure*\}}

\hspace*{1.5cm}\textit{$\backslash$end\{figure*\}}

Figures should be centered and should always have a caption
positioned under it. The font size to use is 9-point. No bold or
italic font style should be used.

\begin{figure}[!h]
  %\vspace{-0.2cm}
  \centering
   {\epsfig{file = SCITEPRESS.eps, width = 5.5cm}}
  \caption{This caption has one line so it is centered.}
  \label{fig:example1}
 \end{figure}

\begin{figure}[!h]
  \vspace{-0.2cm}
  \centering
   {\epsfig{file = SCITEPRESS.eps, width = 5.5cm}}
  \caption{This caption has more than one line so it has to be justified.}
  \label{fig:example2}
  \vspace{-0.1cm}
\end{figure}

The final sentence of a caption should end with a period.



Please note that the word "Figure" is spelled out.

\subsubsection{Equations}

Equations should be placed on a separate line, numbered and
centered.\\The numbers accorded to equations should appear in
consecutive order inside each section or within the contribution,
with the number enclosed in brackets and justified to the right,
starting with the number 1.

Example:

\begin{equation}\label{eq1}
    a=b+c
\end{equation}

\subsubsection{Program Code}\label{subsubsec:program_code}

Program listing or program commands in text should be set in
typewriter form such as Courier New.

Example of a Computer Program in Pascal:

\begin{small}
\begin{verbatim}
 Begin
     Writeln('Hello World!!');
 End.
\end{verbatim}
\end{small}


The text must be aligned to the left and in 9-point type.

\vfill
\subsubsection{Reference Text and Citations}

References and citations should follow the Harvard (Author, date)
System Convention (see the References section in the compiled
manuscript). As example you may consider the citation
\cite{Smith98}. Besides that, all references should be cited in the
text. No numbers with or without brackets should be used to list the
references.

References should be set to 9-point. Citations should be 10-point
font size.

You may check the structure of "example.bib" before constructing the
references.

For more instructions about the references and citations usage
please see the appropriate link at the conference website.

\section{\uppercase{Copyright Form}}

\noindent For the mutual benefit and protection of Authors and
Publishers, it is necessary that Authors provide formal written
Consent to Publish and Transfer of Copyright before publication of
the Book. The signed Consent ensures that the publisher has the
Author's authorization to publish the Contribution.

The copyright form is located on the authors' reserved area.

The printed form should be completed and signed by one author on
behalf of all the other authors, and uploaded through the authors' reserved area. Alternatively, you can send it to the secretariat by e-mail or fax.

\section{\uppercase{Conclusions}}
\label{sec:conclusion}

\noindent Please note that ONLY the files required to compile your paper should be submitted. Previous versions or examples MUST be removed from the compilation directory before submission.

We hope you find the information in this template useful in the preparation of your submission.

\section*{\uppercase{Acknowledgements}}

\noindent If any, should be placed before the references section
without numbering. To do so please use the following command:
\textit{$\backslash$section*\{ACKNOWLEDGEMENTS\}}


\vfill
\bibliographystyle{apalike}
{\small
\bibliography{example}}


\section*{\uppercase{Appendix}}

\noindent If any, the appendix should appear directly after the
references without numbering, and not on a new page. To do so please use the following command:
\textit{$\backslash$section*\{APPENDIX\}}

\vfill
\end{document}

